\documentclass{article}
\usepackage[letterpaper, margin=1in]{geometry}
\usepackage{amsmath,amssymb,amsfonts}
\usepackage{alltt}

\title{Changes/additions to GOMA since last manual update}
\author{GOMA Authors}
\date{\today}

\begin{document}
\maketitle

% Changelog for GOMA.  If you add a new capability to GOMA, or significantly
% change any existing feature, please quickly describe the changes below.  
% Update this document and check it into the repository with your changes, 
% so there will be a running documentation of new features after the GOMA 5.0
% manual.  This is a LaTeX document, so unless you're familiar with LaTeX 
% formatting, keep your documentation to plain text.


The purpose of this file is to provide a place to quickly catalog new capabililities in GOMA for which user-documentation is required.  As of Spring 2011, the goma user manual is in Framemaker and very difficult to access and change and rebuild.    Eventually everything in this file will get added to the user manual.  This file provides developers a place to quickly describe their new capabilties and add the beginnings of a doc page for each.   Several examples are provided. 

Some rules are in order:
\begin{enumerate}
  \item Please identify yourself and the date you changed this file. I know we can extract this info from the svn logs, but it is nice to have it here too. (see examples)
  \item Please don't make major re-arrangements of this file because we like to be able to decipher the svn diffs. 
  \item Please add a syntax line for the input or mat files.   This will make it easy to reconstruct when we migrate this into more permanent documentation. 
  \item Above all, please add your new changes!!! Even if it is terse because you don't have time.   
\end{enumerate}

Finally, NOTE that we have provided several categories for changes.   Specifically, 1) Configuration Management, 2) Goma Input File capabilities, 3) Goma Material file Capabilities and 4) Overall Capability.    Feel free to add more categories.  

%***************************************************
%***********s*******POOR-MAN's ISSUE TRACKING OUTSTANDING ITEMS
%***************************************************

Issue:  CHEMKIN untested for 10 years and current test suite problems are inactive.   
Issue:  mm_input.c: Domain Mapping File is a chemkin thing that needs to be deleted or ressurected.   
%***************************************************
%************Configuration Management Changes*******
%***************************************************
\section{Configuration Management Changes}
\begin{alltt}

PRS 11/29/2012: NOTE ADDED REFERENCES TO OPENMPI, TRILINOS 10 to 6.0 Manual

ADDED already
Capability: Re-worked makefile system
Developer: ebenner  Date 9/1/2010 
Brief description: Re-worked makefile system.  Makefile and Makefile_debug and 
  Makefile_guts are the only makefiles you need now. 
ADDED already
Capability: Build system for supporting Goma TPLs
Developer: ebenner  Date 9/1/2010 
Brief description:  Updated build system for supporting TPLs.   Overall Makefile

ADDED already
Capability: upgrade to Trilinos 10.0
Developer: ebenner  Date 9/1/2010 
Brief description:  Associated changes to goma and makefiles for Trilinos 10.0

ADDED already
Capability: upgrade to openmpi
Developer: sarober  Date 9/1/2010 
Brief description:  Associated changes to goma and makefiles for openmpi

\end{alltt}
%***************************************************
%************Goma Input File Capabilities***********
%***************************************************
\section{Goma Input File Capabilities}
\begin{alltt}


ADDED TO MANUAL 12/2/2012:
Capability: Transient restarts.  
Developer: Unknown  Date: Unknown
Description: A negative restart time will trigger goma to look for the
  nearest restart time value in an exodus file
Example:    Initial Time = -10.0  (see Initial time card).   
  In this example the restart exoII file is searched for the time
  closest to -10.0. 

ADDED TO MANUAL 12/2/2012:
Capability: Transient external field data
Developers: smdavis, prschun  Date: 3/1/2011
Brief Description: A "time_dependent" flag will bring in the nodal variable  
  interpolated to the current time step.   This can be useful in doing 
  transient coupling in problems like advection-diffusion. 
Example Cards: External Field = BEN_1 Q1 infile.exoII timedependent  
  See External Field card description.  The ``timedependent'' modifier
  which is optional triggers this field to be read in time from the
  exodus file infile.exoII and goma calls for the time plan closest to
  its actualy time plane.  

Capability: BC = F_DIODE_BC    Talk to randy Schunk or Scott Roberts. 
Developers: prschun   Date: 5/20/2011
Description: Unique boundary condition for the Fill (LS) function
  which allows liquid to leave a domain, but not come in.  
Example: BC = F_DIODE_BC

ADDED TO MANUAL (12/2/2012)
Capability: Level Set Initialize = variable_type init_value_minus init_value_plus
Developers: ??
Description: No idea but I think it is to inialize other fields around the zero level set
Example: Level Set Intialize Velocity 3.34 0  

ADDED TO MANUAL (12/2/2012)
Capability: VOLUME_INT LUB_LOAD: Lubrication Load post processing capability 
Developers: Randy schunk
Description:  Integrate the lubrication pressure over entire mesh area.  
Example: 
Post Processing Volumetric Integration =
VOLUME_INT = LUB_LOAD 1 0 volume.out
END OF VOLUME_INT

ADDED TO MANUAL (12/2/2012)
Capability:  Post-processing for Lame MU coefficient
Developers:  Scott A. Roberts, sarober
Description:  Adds an additional post processing item for the Lame MU coefficient.
Post Processing Specifications
Lame MU = yes

ADDED TO MANUAL (12/2/2012)
Capability:  Post-processing for Lame LAMBDA coefficient
Developers:  Scott A. Roberts, sarober
Description:  Adds an additional post processing item for the Lame LAMBDA coefficient.
Post Processing Specifications
Lame LAMBDA = yes

ADDED TO MANUAL (12/7/2012)
Capability:  Post-processing for von Mises stress and strain invariants
Developers:  Scott A. Roberts, sarober
Description:  Adds an additional post processing item for the post processing of the von Mises invariants for the strain and stress tensor.
Post Processing Specifications
Von Mises Strain = yes
Von Mises Stress = yes

PRS NOT SURE WHAT TO ADD HERE TO MANUAL. UNDONE.   (12/7/2012)
Capability:  Post-processing of shell normals
Developers:  Scott A. Roberts (sarober)  Date:  2/6/2012
Description:  Overloaded the SHELL_NORMALS post processing to now take the
  value calculated by R_LUBP and output that.
Example:
  Shell Normal Vectors = yes

ADDED TO MANUAL (12/7/2012)
Capability:  Command-line override of 'Initial Time' and 'Maximum Time'
Developers:  Scott A. Roberts (sarober)  Date:  2/27/2012
Description:  Allows setting the simulation initial (start) time and the
  maximum (end) time via a command-line argument.  -ts specifies initial
  time while -te specifies maximum time.
Example:
  ./goma -a -i file.i -ts 0.001 -te 0.1

PRS NOT SURE WHAT TO ADD HERE TO MANUAL. STILL TESTING.   (12/7/2012)
Capability:   Added global variable Porous_liq_inventory to allow for
tracking of the actual current liquid inventory in shell porous
media.  This variable is used in a user-defined boundary conditon to
shut off the inflow of liquid after a finite amount is reached.   See
sample in user_bc.c under shell_p_open_user_surf
Developers: P. R. Schunk Date 7/16/2012
Description: see above
Example: see SH_P_OPEN_USER

PRS NOT SURE WHAT TO ADD HERE TO MANUAL. STILL TESTING.   (12/7/2012)
Capability: Boundary condition for user-defined open pore pore
pressure.   SH_P_OPEN_USER
Developers: P. R. Schunk Date 7/16/2012
Description: see above
Example: BC = SH_P_OPEN_USER {max_inventory_in_porous_media}

ADDED TO MANUAL (12/7/2012)
Capability:  New second layer lubrication equation R_LUBP_2 (LUBP_2)
Developer: P. R. Schunk, August 2012
Description:  Added second reynolds lubrication equation to
accommodate multilayer alternating stacks of lubrication and porous
shells, for multilayer thin porous media.  
Example: EQ = lubp_2             Q1 LUBP_2        Q1         1   1   1

ADDED TO MANUAL (12/7/2012)
Capability:  New second layer thin-porous open cell transport (DARCY)
equation companion to R_SHELL_SAT_OPEN, called  R_SHELL_SAT_OPEN_2
Developer: P. R. Schunk, August 2012
Description:  Added second open porous shell equation to accomodate
2-layer stacks of thin porous structures alternating with Reynolds
lubrication regions. 
Example: EQ = shell_sat_open_2   Q1 SH_P_OPEN_2   Q1 1           1   1

Capability: Boundary condition for hydrostatic pressure
Developer: D. Hariprasad, December 2014
Description: Specifies pressure as P = P_0 + rho*g*x where rho is a vairable density and the user specifies P_0.
Example:  BC = FLOW_PRESSURE_VAR NS 1 0

Capability: Galerkin least-squares continuity stabilization 
Developer: D. Hariprasad, February 2015
Description: Adds a Galerkin least-squares stabilization term using the residual of the continuity equation for incompressible or steady-state compressible flows.  This should help with high Reynolds number flow.  Similar to the Pressure Stabilization already implemented in Goma.  See Franca and Hughes 1988 for an abstract discussion of the method or Forster, Wall, and Ramm 2010 for a Navier-Stokes based discussion of the method.  The choice of a local or global stability parameter is left to the user.
Example: Continuity Stabilization = local


PRS ADDED TO MANUAL (12/11/2012) as noted:

Capability:  Specialized interpolations for sharp-interface tracking. 
/*
 * I_FLAG CONSTANTS
 * ------------------------------------------------------------------------
 *
 * These constants are used in the i[] field in the Problem_Description
 * structure
 *
 * Interpolation functions for variables...
 */

PRS NOT ADDED (4)
#define I_P0_G		19 /* Piecewise constant with ghost values. */
#define I_P1_G		20 /* Piecewise linear with ghost values. */
#define I_Q1_G		21 /* Lagrangian linear with ghost values. */
#define I_Q2_G		22 /* Lagrangian quadratic with ghost
values. */
PRS ADDED (4)
#define I_P0_XV		23 /* Piecewise constant with enrichment for jump in value. */
#define I_P1_XV		24 /* Piecewise linear with enrichment for jump in value. */
#define I_Q1_XV		25 /* Lagrangian linear with enrichment for jump in value. */
#define I_Q2_XV		26 /* Lagrangian quadratic with enrichment for
jump in value. */
PRS NOT ADDED (3)
#define I_P1_XG		27 /* Piecewise linear with enrichment for jump in gradient. */
#define I_Q1_XG		28 /* Lagrangian linear with enrichment for jump in gradient. */
#define I_Q2_XG		29 /* Lagrangian quadratic with enrichment for
jump in gradient. */
PRS ADDED (6)
#define I_P0_GP		30 /* Piecewise constant confined to positive side of interface. */
#define I_P1_GP		31 /* Piecewise linear confined to positive side of interfaces. */
#define I_Q1_GP		32 /* Lagrangian linear confined to positive side of interface. */
#define I_Q2_GP		33 /* Lagrangian quadratic confined to positive side of interface. */
#define I_P0_GN		34 /* Piecewise constant confined to negative side of interface. */
#define I_P1_GN		35 /* Piecewise linear confined to negative side of interfaces. */
#define I_Q1_GN		36 /* Lagrangian linear confined to negative side of interface. */
#define I_Q2_GN		37 /* Lagrangian quadratic confined to
negative side of interface. */
PRS NOT ADDED (7)
#define I_Q1_HG		38 /* Lagrangian linear with discontinuous enrichment for jump in gradient. */
#define I_Q1_HV		39 /* Lagrangian linear with discontinuous enrichment for jump in value. */
#define I_Q1_HVG	40 /* Lagrangian linear with discontinuous enrichment for jump in value and gradient. */
#define I_Q2_HG		41 /* Lagrangian quadratic with discontinuous enrichment for jump in gradient. */
#define I_Q2_HV		42 /* Lagrangian quadratic with discontinuous enrichment for jump in value. */
#define I_Q2_HVG	43 /* Lagrangian quadratic with discontinuous enrichment for jump in value and gradient. */
#define I_TABLE		44 /* Table Interpolation	*/




\end{alltt}
%***************************************************
%************Goma Material File Capabilities********
%***************************************************
\section{Goma Material File Capabilities}
\begin{alltt}

ADDED PRS (12/13/2012)
Capability: LEVEL_SET option for Electrical conductivity model
Developer: Randy Schunk  Date: 5/20/2011
Description:  LEVEL_SET model allows for this property to be changed
  from one phase to another, with the phase being marked
  with a level-set 0. 
Example: Electrical Conductivity	= LEVEL_SET  0.1 .0943 0   
  In this example the value of electrical conductivity is 0.1 on the
  negative side of the interface and 0.943 ont he positive.  

Capability: LEVEL_SET option for all properties on THERMAL, CARREAU, and CARREAU_WLF models
Developer: prschun  Date 5/20/2011
Description: See electrical conductivity card above.
Example:
  ADDED PRS (12/13/2012)
  Low Rate Viscosity	          = LEVEL_SET {mu} {1.0} 0.
  ADDED PRS (12/13/2012)
  Power Law Exponent              = LEVEL_SET {1.0} {0.55} 0.
  ADDED PRS (12/14/2012)
  High Rate Viscosity	          = LEVEL_SET {mu}  {0.067} 0.
  ADDED PRS (12/14/2012)
  Time Constant	                  = LEVEL_SET {0.1} {0.1} 0.
  ADDED PRS (12/14/2012)
  Aexp                            = LEVEL_SET      {2.} {2.} 0.
  ADDED PRS (12/14/2012)
  Thermal Exponent                = LEVEL_SET     {0} {2.05} 0.
  ADDED PRS (12/14/2012)
  Thermal WLF Constant2           = LEVEL_SET     {0.681} {0.681} 0.

 ADDED PRS (12/14/2012)
Capability: JOULE_LS model on shell energy equation Ext_field card

ADDED PRS (12/14/2012)
Capability: POLY_TIME lubrication height function model.
Developer:  Scott Roberts  Date:  7/18/2011
Description:  Allows an arbitrary-order polynomial (in time) as the height function 
  model for the upper height function in lubrication problems.  Called through 
  the material file card
Upper Height Function Constants = POLY_TIME c_0 c_1 c_2 c_3 ...
  where the c_i is the constant in front of the t^i term.  

ADDED PRS (12/17/2012)
Capability:  TANGENTIAL_ROTATE lubrication velocity model
Developer:  sarober  Date:  8/9/2011
Description:  Allows the unique specification of tangential motion in a lubrication 
  shell element.  Previous implementations allowed specification only in terms of 
  coordinate directions, but this can be used to rotate a cylinder.
Example:
  Lower Velocity Function Constants = TANGENTIAL_ROTATE {tx} {ty} {tz} {U1} {U2}
  The t vector must never be normal to the shell.  It is then projected onto the 
  shell, and the U1 velocity is specified in a direction that is the cross product 
  between the shell normal and the t vector.  The U2 velocity is then specified in a 
  direction that is normal to that first velocity.

ADDED PRS (12/17/2012)
Capability:  JOURNAL bearing lubrication height function model.
Developer:  sarober  Date:  8/9/2011
Description:  Allows the height function to simulate a journal bearing.  It is only 
  tested for the case where you are rotating a cylinder that is aligned with the 
  z coordinate and is centered at (0,0).  The parameters are the mean gap thickness 
  and the eccentricity.  
Example:
  Upper Height Function Constants = JOURNAL {Hlub} {ecc}

ADDED TO ALL PERTINENT PROPERTIES (PRS 12/21/2012)
Capability:  TABLE based material parameters
Developers: Tom Baer and Allen Roach  Date:  1998-1999
Description:  Many material properties (e.g. viscosity) are able to be defined through
  a table as a function of another variable, such as temperature.  This is not documented 
  anywhere in the manual.

ADDED PRS (12/21/2012...The eve of the end of the mayan world)
Capability:  TABLE for LAME MU
Developers:  sarober  Date:  10/07/2011
Description:  Specify the LAME MU coefficient through a table as a function of temperature.
Example:  

ADDED PRS (12/17/2012)
Capability:  POISSON_RATIO for LAME LAMBDA
Developers:  sarober  Date:  10/07/2011
Description: If you have a complex model for your Lame mu coefficient already, why repeat 
  it for lambda?  \(\lambda=2\mu\nu/(1-2\nu)\)
Example:  Lame LAMBDA = POISSON_RATIO 0.3

ADDED PRS (12/18/2012)
Capability: ADD TREF reference to BOUSSINESQ and BOUSS models 

Capability:  Lame Temperature Shift = {CONSTANT | POWER_LAW | TABLE}
Developers: prschun  Date: 4/10/2012
Description: Applieds a temperature shift fact to LAME Mu models (right now hardwired to only work with the Lame Mu = TABLE option).   When you use the Lame Mu = TABLE card you must supply the card
	Lame Temperature Shift = CONSTANT {flt}
or
        Lame Temperature Shift = POWER_LAW {Troom} {Tmelt} {exponent}
Example: Lame Temperature Shift = POWER_LAW 25. 550. 0.94

Capability:  Liquid Constitutive Equation = SYLGARD
Developer: RRRAO
Descritpion and problem: temperature dependent polymer concentration model specifically for this
sylgard material.    NOT ADDED TO MANUAL.    

Capability:  Liquid Constitutive Equation = FOAM_EPOXY
Developer: RRRAO
Descritpion and problem: Need to add to goma6.0 manual a description of this model. 
    NOT ADDED TO MANUAL.    

Capability: Dilational Viscosity 
{TABLE ! DILVISCM_KAPPAWIPESMU? | ...}
Developer: RRRao?????
Exmamples???  

Capability: Shell User Parameter = float 1, etc.   In material file
Developer: rbsecor???
Description:  Looks like this was an experimental capability to feed paramters with old lubrication shell technology used by robert secor et al.   Action: Check with him and rip it out if no longer needed.  

Capability: Acoustic Wave Number (material property card??)
Developer: rbsecor  
Examples None

Capability: Acoustic Impedance (material property card??)
Developer: rbsecor  
Examples None

Capability: Acoustic Absorption
Developer: rbsecor  
Examples None

ADDED PRS (12/21/2012)
Capability: Lower Velocity Function Card SLIDER_POLY_TIME
Developers: prschun Date: 10/26/2011
Description: New model for the Lower velocity function card which allows a general time-polynomial.   The first constant is a time-scale factor which can be used to scale the time, and the remainder are the coefficients of the polynomial in time starting from the constant term.  
Example: 
Lower Velocity Function Constants = SLIDER_POLY_TIME   {time_scale = 1000.} {-1.0*100.0*44.170122} {100.*361.41971} {-100.*807.45844} {100*256.08906}  {-100.*35.316517} {100.*2.1560711} {-100.*0.030632551} {-100.*0.0013365347}

ADDED PRS (12/21/2012)
Capability: Upper Velocity Function Card CIRCLE_MELT
Developers: Scott A. Roberts, sarober Date: 03/20/2012
Description: New model for the upper velocity function card which allows a converging or diverging height that is like a circle.  Also works with melting.  First coefficient is the x location of the circle center, second coefficient is the radius, and the third is the minimum height. 
Example: 
Upper Velocity Function Constants = CIRCLE_MELT {x_0=5} {r=1} {hmin=0.001}  

Capability:  Faux plasticity model
Developers:  Scott A. Roberts, sarober, Date:  4/10/2012
Description:  Short of using a full EVP model, it may be possible to ``fake'' plasticity by lowering the lame coefficients once the strain gets into the plastic regime, according to the stress-strain curve of the material.  This new capability allows the user to input the stress-strain curve, and the Lame mu will be calculated from the von Mises stress and strain, applied to the stress strain curve.  Also, must use Poisson Ratio as the model for Lame Lambda so everything is consistent.
Example:
Lame MU       = TABLE 2 FAUX_PLASTIC 0 LINEAR FILE=stress_strain.txt
Lame LAMBDA   = POISSON_RATIO 0.33

ADDED TO MANUAL PRS (12/21/2012)
Capability:  LOWER_DISTANCE model for Lower Height Function Model
Developers:  Scott A. Roberts, sarober
Date:  May 2, 2012
Description:  Allow user to specify an arbitrary lower height function model via a TABLE read that is translated in the x direction with the lower velocity function model.  Requires use of the SLIDER_POLY_TIME lower velocity function model.
Example:
Lower Height Function Constants   = TABLE 2 LOWER_DISTANCE 0 LINEAR FILE=shell.dat
Contents of shell.dat is a table with 2 columns.  The first is a position, the second is a height.

ADDED TO MANUAL PRS (12/22/2012)
Capability:  NEW POROUS CLOSED SHELL and POROUS OPEN SHELL cards
required
Developer: Randy Schunk, prschun
Date June 1, 2012
Description: Hooked up shell porous media to main infrastructure of
porous media equations in goma. This entailed some substantial changes
to the material files.    I attempt to give you a recipee of the
changes required to run open and closed shell porous media problems.
Examples: see below

ADDED TO MANUAL PRS (12/22/2012)
NEW CARDS and Models
Capability: New Media type for shell porous media
POROUS_SHELL_UNSATURATED
Developer: Randy Schunk, prschun
Description: this type forces the material file inputs for porous
properties to be taken from the regular porous media cards. 
Example:
Media Type	 	           = POROUS_SHELL_UNSATURATED

ADDED TO MANUAL PRS (12/22/2012)
Capability: New Saturation model. SHELL_TANH
Developer: Randy Schunk, prschun
Description: This model was written by Scott Roberts. I simply moved
it from the hardwired mode to the input deck.    
Example:
Saturation = SHELL_TANH {sigma=25.} {theta=30.0} {Rmin =0.1*R} {Rmax = 10.*R}

ADDED TO MANUAL PRS (12/22/2012)
Capability:  Rename of Porous Shell Height card
Developer: Randy Schunk, prschun
Description: ``Porous Shell Height'' card is simply a name change
from ``Porous Shell Closed Height'' card.  
Example: 
Porous Shell Height               = CONSTANT 0.00001

ADDED TO MANUAL PRS (12/22/2012)
Capability:  Rename of Porous Shell Radius card
Developer: Randy Schunk, prschun
Description: ``Porous Shell Radius'' card is simply a name change
from ``Porous Shell Closed Radius'' card.  
Example: 
Porous Shell Radius              = CONSTANT 0.00001

ADDED TO MANUAL PRS (12/22/2012)
Capability:  Cross shell permeability kappa card
Developer: Randy Schunk, prschun
Description: Required card for all shell porous media models.  this
card sets the cross-shell permeability. Currently only used for open
porous shells. 
Example: 
Porous Shell Cross Permeability = CONSTANT {20*0.000000819505}

ADDED TO MANUAL PRS (12/22/2012)
Capability:  Permeability  EXTERNAL_FIELD model
Developer: Randy Schunk
Description:  Brings in exoII mapping of porous permeability as computed
from an image-to-mesh capability
Example: Permeability   = EXTERNAL_FIELD {1.0}

ADDED TO MANUAL PRS (12/22/2012)
Capability:  Saturation TANH_EXTERNAL model
Developer: Randy Schunk
Description:  Brings in exoII mapping of porous saturation function as computed
from an image-to-mesh capability. The mapped field is on a 0-1 gray
scale between two regions, and the capillary saturation curves are
scaled linearly between two extremes on this scale. The first 4 floats
are for the first curve, and the second 4 for the second curve.  The
definitions of the floats are given on the TANH saturation model 
Example: Saturation = TANH_EXTERNAL 0.2 0. 2.0 70000.0 0.2 0. 1.1 100000.0


ADDED TO MANUAL PRS (12/22/2012)
Capability:  Cross shell permeability kappa EXTERNAL_FIELD model
Developer: Randy Schunk
Description:  Brings in exoII mapping of this permeability as computed
from an image-to-mesh capability
Example: Porous Shell Cross Permeability   = EXTERNAL_FIELD {1.0}

ADDED TO MANUAL PRS (12/22/2012)
Capability:  Porous Shell Initial Pore Pressure  card
Developer: Randy Schunk, prschun
Description: Required card. Place holder for shell porous media initial pressure.  
Example: 
Porous Shell Initial Pore Pressure = CONSTANT -4000


Recipee for converting all POROUS CLOSED SHELL MODELS (Take out -
 cards, and add + cards):
 Porous Shell Closed Porosity      = CONSTANT 0.1
-Porous Shell Closed Height        = CONSTANT 0.00001
-Porous Shell Closed Radius        = CONSTANT 0.00001
+Porous Shell Height               = CONSTANT 0.00001
+Porous Shell Radius               = CONSTANT 0.00001
 Porous Shell Closed Gas Pressure  = CONSTANT 1.0e6
 Porous Shell Atmospheric Pressure = CONSTANT 1.0e6
 Porous Shell Reference Pressure   = CONSTANT 0.0
+Porous Shell Cross Permeability = CONSTANT {0.}
+Porous Shell Initial Pore Pressure = CONSTANT 0.


Recipee for converting all POROUS OPEN SHELL MODELS  (Take out -
 cards, and add + cards):
 # Porous term terms for linking lubrication to level set field
-Media Type                        = POROUS_BRINKMAN
-Porosity                           = CONSTANT 1.0
-Permeability                              = CONSTANT 1.0
-Brinkman Porosity                  = CONSTANT 1.0
-Brinkman Permeability              = CONSTANT 1.0
-FlowingLiquid Viscosity            = CONSTANT 0.0
-Inertia Coefficient                = CONSTANT 0.0

+Media Type                        = POROUS_SHELL_UNSATURATED  <<<<N. B.!
+Porosity                = CONSTANT    {0.4}
+Permeability            = ORTHOTROPIC {2*0.000000819505} {2*0.000000819505} {2*
0.000000819505} 1. 0. 0. 0. 1. 0. 0. 0. 1. 
 
+Capillary Network Stress = PARTIALLY_WETTING
+Rel Gas Permeability     = CONSTANT  {1./0.01} 
+Rel Liq Permeability     = VAN_GENUCHTEN  0. 0. 0.7 {lqvisc=0.01}
+{R=0.01} 
+Saturation             = SHELL_TANH {sigma=25.} {theta=30.0} {Rmin = 0.1*R} {Rm
ax = 10.*R}
+
+--Misc Porous numerical props (Porous Section 2)
+Porous Weight Function          = SUPG 1.0
+Porous Mass Lumping             = true
+Porous Advective Scaling         = CONSTANT 0 1.
+
+---Porous Species Properties (Porous Section 3)
+/* Darcy fickian only for saturated media */
+Porous Diffusion Constitutive Equation = DARCY_FICKIAN
+Porous Gas Diffusivity              = CONSTANT  0  0.
+
+Porous Latent Heat Vaporization = CONSTANT  0   0.
+Porous Latent Heat Fusion       = CONSTANT  0   0.
+Porous Vapor Pressure           = NON_VOLATILE  0   0.
+Porous Liquid Volume Expansion  = CONSTANT  0   1.
+                                            ^
+***Species Number (all zero for primary liquid solvent)****|
+
+--Porous Gas-inert properties
+/*Note. if you change p_ambient here you need to make sure the initialize
+card and the boundary conditions are compatible with the saturation curve. */
+
+Porous Gas Constants           = IDEAL_GAS   {mwair=0} {rgas=0} {temp=0} {p_amb
ient=0}
+
+
 ####################
 ### Source Terms ###
 ####################
@@ -62,11 +88,13 @@
 ### Porous Shell Terms ###
 ##########################
 Porous Shell Closed Porosity     = CONSTANT 0.1
-Porous Shell Closed Height       = CONSTANT 0.25
-Porous Shell Closed Radius       = CONSTANT 0.01
+Porous Shell Height              = CONSTANT 1.0
+Porous Shell Radius              = CONSTANT 0.01
 Porous Shell Closed Gas Pressure = CONSTANT 1e6
-Porous Shell Atmospheric Pressure = CONSTANT 1e6
+Porous Shell Atmospheric Pressure = CONSTANT 0.
 Porous Shell Reference Pressure   = CONSTANT 1e6
+Porous Shell Cross Permeability = CONSTANT {20*0.000000819505}
+Porous Shell Initial Pore Pressure = CONSTANT -4000

\end{alltt}
%***************************************************
%************Overall Capabilities*******************
%***************************************************
\section{Overall Capabilities}
In this section add brief description of a general capability which has been added which may involve many input cards, etc. Those cards should appear in the above sections.  
\begin{alltt}

ADDED TO MANUAL PRS (12/22/2012)
Capability:  Enable phase fields with shell elements and in parallel.
Developer: Randy Schunk, Aug 2012
Brief description: Phase field was not compatible with parallel
processing (dp_vif, brk/fix, etc.).    Fixed this.   Also enabled
phase field to be used with thin shell lubrication with R_LUBP_2
equation)

ADDED TO MANUAL PRS (12/22/2012). 
Capability: True shell element capability.   
Developer(s): prschun, sarober, tjipdowi, hkmoffa.  Date: 2009-2011
Brief Description.  3D shells you must use a SHELL4 or SHELL9 element type in exodus.   
  Many shell equations added now. Seperate user-manual exists for this.  

ADDED TO MANUAL PRS (12/22/2012)
Capability: Time-dependent exodus II input:   
Developer(s): smdavis, prschun.  Date: 3/1/2011
Brief Description: We have the ability now to bring in transient field data to drive a 
  calculation.   That is, if the external field variable is flagged appropriately, then 
  that information is brought in at every selected time step and interpolated to best 
  match the given time step size.   

ADDED TO MANUAL PRS (12/22/2012)
Capability: Pixel-to-image-to-mesh capability.
Developer(s): prschun, esecor, tjipdowi, sarober Date:8/9/2011
Brief Description:  We can now read in through external fields a pixel map in the format 
  no. pixel points
  x_1 y_1 z_1 value
  x_2 y_2 z_2 value
  ...
  x_N y_n z_n value

ADDED TO MANUAL PRS (12/22/2012)
External Pixel Field = {variable name} {Q1|Q2} pixel.txt {blockid}
With this capability one can read in a pixel field generated from any drawing tool or 
  any image (in this format) and map it onto the mesh for a variety of uses as an 
  external field variable.  More to come. 

Capability:  Maximum-through-time history of von Mises strain in mesh
Developer(s):  Scott A. Roberts, sarober  Date:  4/12/2012
Brief Description:  Calculates the maximum von Mises strain that has been experienced at any given material point.  This was designed to be used for FAUX_PLASTICITY, and will soon be required for its use.
Example:  
EQ = max_strain       Q1 MAX_STRAIN    Q1 1               1

Capability: Documentation and testing of element quality specifications (rd_elem_quality_specs in mm_input.c)
Developers: Randy Lober (circa 1998)
Description: These capabilities are used to ouput and possibly deploy in solution control of element quality during free and moving boundary ALE calculations.  They have really never been used, but the idea wass for adaptivity.  
The issue here is that there use is NOT documented.    Should they be?  Or should they be deprecated.   The cards are:
Jacobian quality weight= yes | TRUE
Volume change quality weight=
Angle quality weight=
Triangle quality weight=
Element quality tolerance type=

\end{alltt}
\end{document}
